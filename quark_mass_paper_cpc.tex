\documentclass[12pt]{article}

\usepackage{amsmath,amssymb,amsfonts}
\usepackage{graphicx}
\usepackage{booktabs}
\usepackage{hyperref}
\usepackage[margin=1in]{geometry}

\title{Scale-Invariant Mass Relation $2(m_d/m_u)^3 = m_s/m_d$ from Computational Analysis of Lattice QCD Data}

\author{A. Brilliant \\
\small Applied Dynamics Research (AD Research) \\
\small Computational Physics Division \\
\small Sapporo, Japan \\
\small ORCID: 0009-0004-8024-5442 \\
\small Senior Member, IEEE \\
\small \texttt{ab@ad-research.org}}

\date{}

\begin{document}
\pagestyle{plain}

\maketitle

\footnotetext{No funding was received for this research.}

\begin{abstract}
We report a scale-invariant relation among light quark masses: $2(m_d/m_u)^3 = m_s/m_d$. This empirical constraint, identified through systematic computational analysis enabled by FLAG 2024's improved precision, is preserved under renormalization group evolution from 0.5 GeV to 1 TeV with deviations $<10^{-5}$. Verification against FLAG 2024 world averages shows agreement within $0.3\sigma$. This relation reduces the three independent light quark mass parameters to a single degree of freedom, enabling prediction of all three masses from one input value—a parameter reduction not previously identified in the light quark sector. The relationship was initially identified in Particle Data Group values and subsequently validated with improved precision in FLAG 2024 lattice results. Monte Carlo statistical analysis with Bonferroni correction yields $p < 0.005$ against random occurrence. The mathematical simplicity and scale invariance of this relationship suggest it may reflect previously unrecognized structure in the flavor sector that could inform theoretical development.
\end{abstract}

\textbf{Keywords:} Quark masses, Light quarks, Renormalization group, Scale invariance, Computational phenomenology, Flavor physics

\textbf{PACS:} 12.15.Ff, 14.65.-q, 12.38.Gc, 11.10.Hi

\section{Introduction}
\label{sec:intro}

The pattern of quark masses represents one of the unexplained puzzles of the Standard Model. While quantum chromodynamics (QCD) successfully describes strong interactions, it provides no predictive framework for the values of quark masses. Understanding whether mathematical structure exists within these masses could provide clues to physics beyond the Standard Model.

Empirical mass relations have historical precedent. The Koide formula for charged leptons has maintained sustained accuracy for decades, though its origin remains unexplained. While scale dependence of such relations has been viewed as problematic, this may instead indicate that certain energy scales are particularly revealing of underlying structure.

Recent advances in lattice QCD, particularly the FLAG 2024 world averages with reduced uncertainties, provide an opportunity to systematically explore mathematical relationships among quark masses with improved precision. In the absence of first-principles theoretical guidance for mass generation, computational analysis of precise lattice determinations offers a logical approach to identifying empirical patterns that may reflect underlying structure. The precision of contemporary lattice results, with uncertainties at the 2-4\% level for light quark mass ratios, enables meaningful tests of relationships involving simple mathematical forms—an analysis that was previously limited by larger experimental uncertainties.

In this work, we adopt such a computational analysis approach to systematically explore patterns in lattice QCD determinations of quark masses. Recent progress by the Flavour Lattice Averaging Group (FLAG) and individual lattice collaborations (BMW, MILC, HPQCD, ETM) has provided increasingly precise mass determinations, enabling quantitative tests of mathematical relationships. We report a scale-invariant relationship among light quark masses: $2(m_d/m_u)^3 = m_s/m_d$. This constraint reduces the number of free parameters in the light quark sector from three to one, a significant simplification that suggests underlying structure. We verify its consistency across energy scales using QCD renormalization group evolution and validate it against FLAG 2024 lattice QCD data. The identification of such mathematically clean relations may inform the development of flavor models and could be particularly relevant for the vibrant lattice QCD community. While the physical mechanism remains unclear, we discuss possible interpretations and invite theoretical development.

\section{Theoretical Background}

The renormalization group equations for QCD are:

\begin{equation}
\frac{d\alpha_s}{d\ln\mu} = \beta(\alpha_s),\quad \frac{dm_i}{d\ln\mu} = \gamma_i(\alpha_s)m_i
\end{equation}

For mass ratios $R_{ij} = m_i/m_j$:

\begin{equation}
\frac{dR_{ij}}{d\ln\mu} = [\gamma_i(\alpha_s) - \gamma_j(\alpha_s)]R_{ij}
\end{equation}

To leading order, all quarks have the same anomalous dimension $\gamma_m = -8\alpha_s/(3\pi)$, making mass ratios approximately scale-invariant. This is standard QCD; our contribution is identifying specific numerical values these ratios take.

\section{Methods}

\subsection{Computational Phenomenology Approach}

The precision of contemporary lattice QCD determinations, exemplified by FLAG 2024's reduced uncertainties, enables systematic computational searches for mathematical structure in quark mass ratios. With current uncertainties at the 2-4\% level for light quark mass ratios, relationships involving simple mathematical forms can be meaningfully tested—a capability that represents significant progress over previous decades where larger uncertainties limited discriminatory power.

Given the unique status of quark masses—fundamental parameters lacking first-principles derivation within the Standard Model—we adopt an explicitly exploratory computational approach. Our methodology deliberately:

\begin{enumerate}
\item \textbf{Creates no new mathematics} - We search only for patterns using existing mathematical operations and established number theory
\item \textbf{Assumes no new physics} - All analysis respects established QCD and Standard Model constraints
\item \textbf{Makes no claims of fundamentality} - We present observations for theoretical and experimental evaluation
\end{enumerate}

This approach acknowledges that in the absence of theoretical guidance, systematic pattern searching represents an essential first step in identifying empirical constraints that can guide theoretical development.

\subsection{Pattern Search Constraints}

To distinguish physical relationships from numerical coincidences, we imposed strict constraints:

\textbf{Theoretical constraints:}
\begin{itemize}
\item No modifications to Standard Model required
\item Consistent with known RG evolution
\item Dimensionless ratios only for fundamental relationships
\end{itemize}

\textbf{Mathematical constraints:}
\begin{itemize}
\item Single mathematical operation per relationship
\item Integer constants $\leq 50$
\item Simple fractions preferred over complex ratios
\end{itemize}

The choice to limit integer constants to 50 prevents overfitting through artificially precise fractions. Large numerators or denominators can trivially approximate any decimal value to arbitrary precision (e.g., using fractions with thousands of digits), but such relationships lack physical meaning. This constraint ensures that identified relationships represent genuine simplicity rather than numerical artifacts, while allowing sufficient flexibility to capture physically meaningful ratios.

\subsection{Analysis Procedure}

We examined quark mass ratios across energy scales from 0.5 GeV to 1 TeV using MILC collaboration QCD renormalization group evolution algorithms. Our computational workflow proceeded as:

\begin{enumerate}
\item Initial pattern identification using Particle Data Group (PDG 2024) values
\item Independent validation against FLAG 2024 data (released subsequently)
\item Scale invariance verification across three orders of magnitude in energy
\item Statistical significance assessment via Monte Carlo methods
\end{enumerate}

The analysis revealed a scale-invariant constraint relating the dimensionless mass ratios:

\begin{equation}
2(m_d/m_u)^3 = m_s/m_d
\end{equation}

The identification sequence is documented: initial observation of patterns in PDG data led to precise formulation, followed by confirmation in FLAG 2024 with improved precision.

Analysis scripts and verification code are available at:

\url{https://github.com/AndBrilliant/Light-Quark-Mass-Ratios}

Timestamped development history is available upon request.

\subsection{Dimensionless Ratios and Unit Independence}

All relationships involve dimensionless mass ratios, independent of unit system. The ratio $m_s/m_d \approx 20$ has the same value in natural units, Planck units, or any other system. Since mass ratios are scale-free quantities, their numerical values have no connection to human measurement conventions.

\subsection{QCD Renormalization Group Evolution}

We implemented MILC collaboration mass running algorithms, solving the coupled differential equations:

\begin{equation}
\frac{dm_i}{d\ln\mu} = \gamma_i(\alpha_s(\mu))m_i,\quad \frac{d\alpha_s}{d\ln\mu} = \beta(\alpha_s)
\end{equation}

with anomalous dimension $\gamma_i(\alpha_s) = -\frac{\alpha_s}{\pi}\left(1 + \frac{\alpha_s}{4\pi}...\right)$ to two-loop order, including flavor threshold matching and 200-step numerical integration.

\subsection{Data Sources}

Analysis began with Particle Data Group (PDG 2024) values where patterns were initially observed. FLAG 2024 world averages provided independent validation, with results cross-checked against individual lattice collaborations (ETM, BMW, MILC, HPQCD). The temporal sequence—identification in PDG followed by improved agreement in FLAG—provides natural protection against overfitting to any single dataset.

\section{Results}

\subsection{Empirical Relationships at $\mu = 2$ GeV}

Computational analysis reveals algebraic structure in the light quark sector:

\begin{table}[h!]
\centering
\caption{Dimensionless algebraic constraints at $\mu = 2$ GeV}
\begin{tabular}{lcccc}
\toprule
Relationship & Predicted & FLAG 2024 & Error & Significance \\
\midrule
$m_s/m_d = 20$ & 20.00 & $20.0 \pm 0.9$~\cite{FLAG2024} & 0.0\% & 0.00$\sigma$ \\
$m_d/m_u = \sqrt[3]{10}$ & 2.154 & $2.162 \pm 0.050$~\cite{FLAG2024} & 0.4\% & 0.16$\sigma$ \\
$m_s/m_u = 2 \times 10^{4/3}$ & 43.09 & $43.66 \pm 2.0$~\cite{FLAG2024} & 1.3\% & 0.29$\sigma$ \\
\bottomrule
\end{tabular}
\end{table}

\begin{table}[h!]
\centering
\caption{Pattern validation: Identification in PDG, confirmation in FLAG}
\begin{tabular}{lccc}
\toprule
Data Source & $m_d/m_u$ & Uncertainty & Deviation from $\sqrt[3]{10}$ \\
\midrule
PDG 2024 (initial) & 2.16 & $\pm 0.10$ & 0.06$\sigma$ \\
FLAG 2024 (validation) & 2.162 & $\pm 0.050$ & 0.16$\sigma$ \\
\bottomrule
\end{tabular}
\end{table}

The improved precision in FLAG 2024, with uncertainties reduced by half, allows more stringent testing. The persistence of patterns across independent datasets with different systematic uncertainties provides evidence against pure numerical coincidence.

\subsection{Scale-Invariant Formulation}

The relationships combine into: $2(m_d/m_u)^3 = m_s/m_d$

This scale-invariant form is preserved under QCD evolution, as verified computationally across three orders of magnitude in energy scale.

\subsection{Verification Across Energy Scales}

\begin{table}[h!]
\centering
\caption{Computational verification of scale invariance}
\begin{tabular}{lccc}
\toprule
Energy Scale (GeV) & $m_d/m_u$ & $m_s/m_d$ & Max Deviation \\
\midrule
0.5 & 2.15444 & 20.0000 & $<10^{-5}$ \\
2.0 & 2.15443 & 20.0000 & $<10^{-5}$ \\
91.2 (Z mass) & 2.15443 & 20.0000 & $<10^{-5}$ \\
173.0 (top mass) & 2.15443 & 20.0000 & $<10^{-5}$ \\
1000.0 & 2.15443 & 20.0000 & $<10^{-5}$ \\
\bottomrule
\end{tabular}
\end{table}

The computational finding of scale invariance distinguishes these patterns from other empirical mass formulas. If these relationships reflect physical structure rather than coincidence, future lattice calculations should asymptotically approach $m_d/m_u = 2.15443...$ ($\sqrt[3]{10}$) rather than merely refining values near 2.16—a falsifiable prediction.

\section{Statistical Assessment}

\subsection{Computational Pattern Significance}

To assess whether the observed relationships are statistically notable, we generated $10^6$ random quark mass sets satisfying only hierarchy constraints: $m_u < m_d \ll m_s$ with bounds $1 < m_d/m_u < 3$ and $15 < m_s/m_d < 25$.

For each random set, we tested whether it satisfied our pattern search criteria: single mathematical operations with integer constants $\leq 50$, achieving agreement within $0.5\sigma$.

\textbf{Results:} Among $10^6$ random mass sets, only 87 produced relationships of comparable simplicity (defined as $\geq 3$ independent relationships within $0.3\sigma$).

\textbf{Uncorrected probability:} $p = 87/10^6 = 8.7 \times 10^{-5}$

\textbf{Multiple testing correction:} Using Bonferroni correction for $N = 50$ effective comparisons across energy scales and mathematical forms:

\begin{equation}
p_{\text{corrected}} = \min(50 \times 8.7 \times 10^{-5}, 1) = 4.35 \times 10^{-3}
\end{equation}

This indicates $p < 0.005$ probability that random mass distributions produce comparable patterns.

\textbf{Caveat:} This analysis demonstrates that random mass distributions do not typically produce such simple relationships. It does not prove the patterns are fundamental - they remain empirical observations requiring theoretical explanation.

\subsection{Cross-Collaboration Consistency}

\begin{table}[h!]
\centering
\caption{Pattern consistency across independent lattice collaborations}
\begin{tabular}{lccc}
\toprule
Collaboration & $m_d/m_u$ & Uncertainty & Deviation from $\sqrt[3]{10}$ \\
\midrule
ETM & 2.15 & $\pm 0.08$ & 0.05$\sigma$ \\
BMW & 2.17 & $\pm 0.07$ & 0.23$\sigma$ \\
MILC & 2.15 & $\pm 0.08$ & 0.05$\sigma$ \\
HPQCD & 2.14 & $\pm 0.06$ & 0.24$\sigma$ \\
FLAG Average & 2.162 & $\pm 0.050$ & 0.16$\sigma$ \\
\bottomrule
\end{tabular}
\end{table}

The consistency across independent computational methods using different lattice actions, volumes, and analysis techniques provides robustness against systematic errors in any single approach.

\section{Discussion}

\subsection{Summary of Findings}

The scale-invariant constraint $2(m_d/m_u)^3 = m_s/m_d$ is preserved under QCD RG evolution with deviations $<10^{-5}$ from 0.5 GeV to 1 TeV. At the standard lattice QCD reference scale of $\mu = 2$ GeV (used by FLAG and individual lattice collaborations for world average determinations), the constraint yields $m_d/m_u = \sqrt[3]{10}$ and $m_s/m_d = 20$, agreeing with data within 0.16$\sigma$ and 0.00$\sigma$ respectively.

\subsection{Possible Interpretations}

\textbf{Scale Invariance from QCD:} The observed scale invariance is consistent with QCD predictions. To leading order, all quarks have the same anomalous dimension, making mass ratios approximately scale-independent. The specific numerical values ($\sqrt[3]{10}$ and $20$) are what require explanation.

\textbf{Flavor Symmetry Breaking:} The relationships may encode information about how flavor symmetry is broken. The appearance of simple algebraic forms suggests possible geometric organization in flavor space.

\textbf{Numerical Coincidence:} We cannot exclude the possibility that these are coincidences within current experimental precision. Future lattice QCD calculations with reduced uncertainties will test this.

\subsection{Constraints on Yukawa Couplings}

The empirical relation implies:
\begin{equation}
2y_d^4 = y_u^3 y_s
\end{equation}

for Yukawa couplings, a non-standard texture not found in conventional flavor models.

\subsection{Limitations}

\begin{enumerate}
\item \textbf{No Theoretical Derivation}: We have not derived these relationships from first principles. However, no first-principles theory of quark mass generation currently exists within or beyond the Standard Model. In the absence of predictive theoretical frameworks, empirical constraints—even when their origin is unclear—can provide valuable guidance for theoretical development. The Koide formula for charged leptons, despite lacking theoretical explanation for over four decades, has informed numerous theoretical investigations into flavor structure.

\item \textbf{Post-Hoc Identification}: The relationships were observed in existing data rather than predicted a priori. However, the temporal sequence of identification in PDG data followed by confirmation in FLAG 2024 with improved precision provides some protection against overfitting. These are empirical observations requiring theoretical explanation, and we hope they will be of interest to researchers in QCD theory and flavor physics.

\item \textbf{Physical Mechanism}: The origin of these patterns remains unclear.
\end{enumerate}

\section{Conclusion}

We report a scale-invariant relationship among light quark masses: $2(m_d/m_u)^3 = m_s/m_d$. This constraint, preserved under QCD renormalization group evolution from 0.5 GeV to 1 TeV with deviations $<10^{-5}$, yields at $\mu=2$ GeV the predictions $m_d/m_u = \sqrt[3]{10}$ (0.16$\sigma$ from FLAG 2024 data) and $m_s/m_d = 20$ (exact agreement).

Monte Carlo analysis with Bonferroni correction yields $p < 0.005$, indicating these patterns are statistically unlikely in random mass distributions. The relationships are consistent across multiple independent lattice collaborations.

These patterns reduce the three light quark masses to one fundamental parameter. The physical origin remains unclear. Possible explanations include emergent properties of flavor symmetry breaking or numerical coincidence within current precision.

We present these findings as empirical observations warranting theoretical investigation rather than established physical law. Future work should focus on: (1) deriving these relationships from first principles, (2) testing validity as lattice precision improves, and (3) exploring implications for Yukawa coupling structure.

\section*{Declaration of Competing Interest}
The author declares no competing financial interests.

\section*{Data Availability}
All data from FLAG 2024 and PDG 2024 public releases. Analysis code and verification scripts available at:

\url{https://github.com/AndBrilliant/Light-Quark-Mass-Ratios}

Timestamped development history available upon request.

\begin{thebibliography}{99}

\bibitem{FLAG2024} Y. Aoki et al. (FLAG Working Group), Eur. Phys. J. C \textbf{84}, 1263 (2024).
\bibitem{PDG2024} S. Navas et al. (Particle Data Group), Phys. Rev. D \textbf{110}, 030001 (2024).
\bibitem{Bazavov2018} A. Bazavov et al. (MILC), Phys. Rev. D \textbf{98}, 054517 (2018).
\bibitem{Blum2016} T. Blum et al. (RBC/UKQCD), Phys. Rev. D \textbf{93}, 074505 (2016).
\bibitem{Borsanyi2015} S. Borsányi et al., Science \textbf{347}, 1452 (2015).
\bibitem{Chakraborty2015} B. Chakraborty et al. (HPQCD), Phys. Rev. D \textbf{91}, 054508 (2015).

\end{thebibliography}

\end{document}